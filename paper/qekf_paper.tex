%%%%%%%%%%%%%%%%%%%%%%%%%%%%%%%%%%%%%%%%%%%%%%%%%%%%%%%%%%%%%%%%%%%%%%%%%%%%%%%%
%2345678901234567890123456789012345678901234567890123456789012345678901234567890
%        1         2         3         4         5         6         7         8

\documentclass[letterpaper, 10 pt, conference]{ieeeconf}  % Comment this line out if you need a4paper

%\documentclass[a4paper, 10pt, conference]{ieeeconf}      % Use this line for a4 paper

\IEEEoverridecommandlockouts                              % This command is only needed if
                                                          % you want to use the \thanks command

\overrideIEEEmargins                                      % Needed to meet printer requirements.

%In case you encounter the following error:
%Error 1010 The PDF file may be corrupt (unable to open PDF file) OR
%Error 1000 An error occurred while parsing a contents stream. Unable to analyze the PDF file.
%This is a known problem with pdfLaTeX conversion filter. The file cannot be opened with acrobat reader
%Please use one of the alternatives below to circumvent this error by uncommenting one or the other
%\pdfobjcompresslevel=0
%\pdfminorversion=4

% See the \addtolength command later in the file to balance the column lengths
% on the last page of the document

% The following packages can be found on http:\\www.ctan.org
%\usepackage{graphicx}
%\usepackage{graphics} % for pdf, bitmapped graphics files
%\usepackage{epsfig} % for postscript graphics files
\usepackage{mathptmx} % assumes new font selection scheme installed
%\usepackage{times} % assumes new font selection scheme installed
\usepackage{amsmath} % assumes amsmath package installed
\usepackage{amssymb}  % assumes amsmath package installed
\newcommand{\transpose}[1]{\ensuremath{#1^{\scriptscriptstyle T}}}
\title{\LARGE \bf
Quaternion EKF
}


\author{Albert Author$^{1}$ and Bernard D. Researcher$^{2}$% <-this % stops a space
\thanks{*This work was not supported by any organization}% <-this % stops a space
\thanks{$^{1}$Albert Author is with Faculty of Electrical Engineering, Mathematics and Computer Science,
        University of Twente, 7500 AE Enschede, The Netherlands
        {\tt\small albert.author@papercept.net}}%
\thanks{$^{2}$Bernard D. Researcheris with the Department of Electrical Engineering, Wright State University,
        Dayton, OH 45435, USA
        {\tt\small b.d.researcher@ieee.org}}%
}


\begin{document}



\maketitle
\thispagestyle{empty}
\pagestyle{empty}


%%%%%%%%%%%%%%%%%%%%%%%%%%%%%%%%%%%%%%%%%%%%%%%%%%%%%%%%%%%%%%%%%%%%%%%%%%%%%%%%
\begin{abstract}

This electro

\end{abstract}


%%%%%%%%%%%%%%%%%%%%%%%%%%%%%%%%%%%%%%%%%%%%%%%%%%%%%%%%%%%%%%%%%%%%%%%%%%%%%%%%
\section{INTRODUCTION}

This template pro

\subsection{Notation}

For clarity, we decided to dedicate this section to


Vectors are represented by normal font variables, i.e. $a,x,y,z$.
Matrices are represented by UPPERCASE letters, i.e. $\mathbf{A,X,Y,Z}$.
Approximated vectors are denoted with a \textit{tilde} on top
to signify its \textit{approximation} designation, i.e.  $\widetilde{a},\widetilde{x},\widetilde{y},\widetilde{z}$. For clarity, we define approximate value representation
in \ref{eq:1} as the original function with added bias and uniform distribution noise.
This form representation is used to represent high-resolution measurements that
are used as approximated input to the system. This is an essential part the EKF
implementation which we describe in detail in later sections.
Estimated vectors are represented with a \textit{hat} on top
to signify its \textit{estimation} designation, i.e.  $\widehat{a},\widehat{x},\widehat{Y},\widehat{Z}$.




\section{OBSERVATION AND ESTIMATION MODELS}

\subsection{Inertial Model}

We start with a Newtonian dynamic model, where the system is described by forces
acting on a rigid body.


For this paper, we decided to use linear and angular velocities as known input
to the system, where we define $u$ as the following.

\begin{equation}
\label{eq:1}
\transpose{u} := \left[ \transpose{v} \transpose{w} \right],
\end{equation}

\begin{equation}
\label{eq:2}
u_{B} = {C}^{I}_{B} u,
\end{equation}


Where matrix ${C}^{I}_{B}$ represents the orthanormal rotation from \textit{inertial frame},
\textit{I}, to robots \textit{body frame}, \textit{B}.



\begin{equation}
\label{eq:3}
\widetilde{v}= v + n_{v} + b_{v},
\end{equation}


\begin{equation}
\label{eq:4}
\widetilde{\omega} = \omega + n_{\omega} + b_{\omega}.
\end{equation}

where $n$ and $b$ represent a normal distribution noise and bias added to the







\subsection{Observation State Definition}

The measurement state definition is defined by linear position, $r$, linear
velocity, $v$, angular velocity, $\omega$, and angular orientation in the quaternion
space, $q$. The observation state vector, $z$, is defined as the following:

\begin{equation}
\label{eq:4}
\transpose{z} :=  \left[\transpose{r}~\transpose{v}~\transpose{\omega}~\transpose{q_{xyzw}} \right],
\end{equation}


Where the observed the quaternion state is used to compute the corresponding state
rotation matrix, ${C}^{I}_{B}$.
It is important to note that the observation data is treated as the groundtruth (Keep?????).




\subsection{EKF Model}

To deploy a modified Kalman filter, we start with the assumption of continuous-time nonlinear system described by the following:

\begin{equation}
\dot{x} = f(x,u),
\end{equation}
\begin{equation}
y = h(x,u).
\end{equation}

Where $f()$ represents the \textit{process} model and $h()$ represent the
\textit{observation} model.
%Variables $\omega_{f}$ and $\omega_{h}$ represent the \textit{process} and \textit{observation} noise, respectively.
Vector $u$ represent input to the system.

\subsection{Estimation State Definition}

The estimation state is defined by the robot's linear position, $r$, and velocity,
$v$, and the body frame orientation in the quaternion space.

\begin{equation}
\label{eq:5}
\transpose{x} :=  \left[\transpose{r}~\transpose{v}~\transpose{q_{xyz}} \right]
\end{equation}

\begin{equation}
\label{eq:6}
P := Cov(\delta x),
\end{equation}

Estimation residual is denoted by $\delta x$


\begin{equation}
\label{eq:7}
\delta \transpose{x} = \left[\transpose{\delta r} ~\transpose{\delta v} ~\transpose{\delta \phi} \right]
\end{equation}

How was $ \transpose{\delta \phi} $ obtained?


\subsection{Estimation Model}

\begin{equation}
\label{eq:8}
\hat{x} =
\end{equation}



\section{QUATERNION ALGEBRA}

Quaternion intro paragraph
Quaternion space is a non-minimal representation belonging to \textit{SO(3)} Lie group.

\subsection{Unit Quaternion}
Moreover, the quaternion term from the dataset has \textit{four terms} with $xyzw$
format.
Hamilton's quaternion defined by 3 perpendicular imaginary axes $i,j,k$ with
real scalars $x,y,z$ and a real term $w$ which constraints other 3 dimension to
a \textit{unit magnitude}. Thus, the fourth term normalizes the vector's magnitude
conveniently and preserves the 3D rotation (3 DOF). We define \textbf{Unit Hamiltonian}
or \textbf{Unit Quaternion} as,


\begin{eqnarray}\nonumber
\label{eq:9}
\mathbb{H}^{1} &:=&\left\{ q_{wxyz}=w+xi+yj+zk~\in \mathbb{H}~| \right.\\
                   && \left.~w^{2}+x^{2}+y^{2}+z^{2}=1 \right\}
\end{eqnarray}
% page 27, sec 2.4.2 of "Quaternion Algebra"

Where superscript 1 in $\mathbb{H}^{1}$ denotes a unit quaternion space with 4
terms. There are two equal representations for $\mathbb{H}^{1}$ subgroup; thus,
we provide a concise definition and notation for both to avoid confusion. The
the first representation is shown in \ref{eq:5} where the four terms of the
quaternion are arranged in $wxyz$ order and it is represented by $q_{wxyz}$.
The second quaternion is arranged in $xyzw$ format and is represented by $q_{xyzw}$.
It is important to note the difference as both are used in our derivation and
implementation.

\begin{equation}
\label{eq:10}
q_{wxyz} = q_{xyzw} ~; ~~~ q_{wxyz},~q_{xyzw} \in \mathbb{H}^{1}
\end{equation}


\subsection{Pure Quaternion}
As previously mention, the three imaginary terms of the quaternion represent the
angles of interest in 3D and the fourth dimension constraints the magnitude. Thus
to avoid computational errors, we use quaternion only with its three imaginary
terms, $xyz$.
This quaternion space representation is defined by $\mathbb{H}^{0}$ and
denoted by $q_xyz$ variables.



\begin{eqnarray}\nonumber
\label{eq:11}
\mathbb{H}^{0} &:=& \left\{ q_{xyz}=xi+yj+zk~ \in \mathbb{H}~| \right.\\
&& \left. ~x,y,z \in \mathbb{R} \right\} \backsimeq  \mathbb{R}^3
\end{eqnarray}
% page 27, sec 2.4.2 of "Quaternion Algebra"



\subsection{Exponential Map}

For calculating incremental rotation in

Incremental rotation estimation using the skew-symmetric matrix obtained form
the rotational rate vector and matrix exponential mapping function, [QEKF01].
Gamma, $\Gamma$, represents incremental

\begin{equation}
\label{eq:}
\Gamma_{0} := \sum_{i=0}^{\infty} \frac{ \left( \Delta t^{i+n}  \right)}{ \\
\left( i + n \right) \! } \\
\omega^{\times i},
\end{equation}

Where $(.)^{\times}$ represents skew-symmetry matrix of a vector


\subsection{Updating Quaternion State}
\begin{equation}
\label{eq:12}
q_{i+1} = \delta q_{i} \otimes \widehat{q}_{i}
\end{equation}



\subsection{Capturing Quaternion Error}
We use the mapping function $\zeta(.)$ to calculate the quaternion
state error from the error rotation vector, [QEKF01].

\begin{equation}
\label{eq:13}
\delta q = \zeta(\delta \phi),
\end{equation}

\begin{equation}
\label{eq:14}
\zeta : v \rightarrow \zeta(v) =
        \begin{bmatrix}
        \sin(\frac{1}{2}\|v\|) \frac{v}{\|v\|} \\
        \cos(\frac{1}{2}\|v\|)
        \end{bmatrix}
\end{equation}



\section{MATH}

Before you


Diagonal matrix Q is in $dim_x \times dim_x$ dimensions and represent the process
noise tolerance or innovation (if I recall correctly--------).

\begin{equation}
%\label{eq:}
Q_{c} =
        \begin{bmatrix}
                Q_{t}
        \end{bmatrix}
\end{equation}





%\begin{equation}
%\label{eq:}

%\end{equation}


Finally, comp

\subsection{Abbreviations and Acronyms}
Defib

\subsection{Units}



\subsection{Equations}

The equations



Note

\subsection{Some Common Mistakes}

\section{USING THE TEMPLATE}

Use this sample docu

\subsection{Headings, etc}

Text heads organiz

\subsection{Figures and Tables}

Positioning Figure



\begin{table}[h]
\caption{An Example of a Table}
\label{table_example}
\begin{center}
\begin{tabular}{|c||c|}
\hline
One & Two\\
\hline
Three & Four\\
\hline
\end{tabular}
\end{center}
\end{table}


   \begin{figure}[thpb]
      \centering
      \framebox{\parbox{3in}{We suggest }}
      %\includegraphics[scale=1.0]{figurefile}
      \caption{Inductance ofd}
      \label{figurelabel}
   \end{figure}




\section{CONCLUSIONS}

A conclu

\addtolength{\textheight}{-12cm}   % This command serves to balance the column lengths
                                  % on the last page of the document manually. It shortens
                                  % the textheight of the last page by a suitable amount.
                                  % This command does not take effect until the next page
                                  % so it should come on the page before the last. Make
                                  % sure that you do not shorten the textheight too much.

%%%%%%%%%%%%%%%%%%%%%%%%%%%%%%%%%%%%%%%%%%%%%%%%%%%%%%%%%%%%%%%%%%%%%%%%%%%%%%%%



%%%%%%%%%%%%%%%%%%%%%%%%%%%%%%%%%%%%%%%%%%%%%%%%%%%%%%%%%%%%%%%%%%%%%%%%%%%%%%%%



%%%%%%%%%%%%%%%%%%%%%%%%%%%%%%%%%%%%%%%%%%%%%%%%%%%%%%%%%%%%%%%%%%%%%%%%%%%%%%%%
\section*{APPENDIX}

Appendixes should appear before the acknowledgment.

\section*{ACKNOWLEDGMENT}

The preferr



%%%%%%%%%%%%%%%%%%%%%%%%%%%%%%%%%%%%%%%%%%%%%%%%%%%%%%%%%%%%%%%%%%%%%%%%%%%%%%%%

References are
\begin{thebibliography}{99}

\bibitem{c1} G. O. Young, Synthetic structure of industrial plastics (Book style with paper title and editor), 	in Plastics, 2nd ed. vol. 3, J. Peters, Ed.  New York: McGraw-Hill, 1964, pp. 1564.
\bibitem{c2} W.-K. Chen, Linear Networks and Systems (Book style).	Belmont, CA: Wadsworth, 1993, pp. 123135.
\bibitem{c3} H. Poor, An Introduction to Signal Detection and Estimation.   New York: Springer-Verlag, 1985, ch. 4.




\end{thebibliography}




\end{document}
